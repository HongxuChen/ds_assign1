\section{Limitations}

Our partition algorithm performance experiment bears the following limitations.
First, the dataset used in the experiment is retrieved from snap, which provides the so called ego network data. The dataset only represent a small portion of the social network starting from one node and traverse through all of its neighbours. Even the combination of several ego network cannot form a real social network. A real social network map will consist many isolated nodes due to inactive account. By considering the real network layout, in the experiment, will produce more accurate results. 

Second, unlike large social networking site, small companies’ user social network may not exhibit some features such as the community structure among the users. An example is that a mobile online game application company may provide social networking services where the player can make friend in the game. However, the player-to-player interaction is limited to between two players rather than in a group, so that one would like to make friends with someone he does not know and all his friends do not know each other. Therefore, the overall network map will be sparse and even distributed among all the nodes. In this case it is difficult to find a community structure in the networks. In our experiment, the dataset is retrieved from facebook and twitter, it may be favourable for some algorithms while make others in disadvantages. In future works, to achieve an unbiased experiment, the user data from small companies needs to be obtained and servers as the input for the testing.

Last, our experiment only considers to store one copy of the user profile into the server.  It is plausible since we are focusing on small business, where the availability of the social networking services may not be a key issue. However, in the real world, some of the companies would like to store the data in several copies. The difference in the number of replication will affect the performance results, for some of the overhead of an algorithm may be considered as a replication stored in a different server. Therefore, the overall overhead decreases when more replication are needed. Future works may take the replication into consideration and get a better performance comparison. 