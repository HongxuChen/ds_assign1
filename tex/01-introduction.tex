\section{Introduction}

Social media has gained its popularity over the years. According to Pew research center\footnote{\url{http://www.pewinternet.org/data-trend/social-media/social-media-use-all-users}}, $74\%$ of all the Internet users have used the social network sites; and the number keeps increasing. 

Due to the need from the customers, in recent years, many startup companies begin to develop their business in this area. Based on the report of Gust\footnote{\url{https://gust.com/startup-trends}}, a company aims to build the connection between startup and investor, in 2015 second quarter, the number of startups has increased by $20\%$ compared to data at the same time in 2014 and $26\%$ for the first quarter of 2015. Among all the startups, $13\%$ of them mainly focus on providing Internet web services and others, like entertainment applications startups may also include social networking functionalities. 

One of the problems in developing software with social network capability is to divide the user data into groups and store them distributedly. A good way of storing the data will improve the availability by providing replication of the data into different server, and the scalability by maintaining the data locality within the server. Many researchers have developed various complexed partition algorithms to improve the performance of the large networking site. However, few of them have optimised their solution to suit the case where a relatively small network is presented and little resource is available. Therefore, this study is going to evaluate a few well-known partition algorithms' performance for the small startup companies.

\bigskip

The rest of the report is organised in the following way: In section \ref{sec:sec2}, a real world problem is defined with high motivation stated. The next section presents a detailed description of the methodology used in the study. Then the experiment setup and implementation are shown. The results and limitation of the experiment are discussed in section \ref{sec:sec4} and \ref{sec:sec5} respectively. The last section concludes this report by highlight the findings and insights.