\section{Problem Description} \label{sec:sec2}


\subsection{Motivation}

Nowadays, more and more small companies have developed software and mobile applications with networking services. Some of them provide networking function as their main business; while others use it as a method to increase the interaction among their users, and thus attracting more potential users of their products. Those small companies need to store their user profiles into the server. However, many startups may encounter problems where they may be lack of budget to deploy adequate number of servers and other infrastructures. Therefore, they must design their storage algorithm wisely in order to fully utilise their limited resource and achieve a relatively good performance. 

One of the aspects which needs considering is to choose an efficient and effective graph partitioning algorithm to separate the user data and deploy them into different servers. It is important because a good deployment of user data at different server can speed up the query process when user would like to obtain it. Small companies need to provide better user experience in order to survive and thrive. Thus, providing user experience with a fast access rate can help the company to maintain and develop its user base.  

\subsection{Problem Definition}

Our experiment aims to find a suitable graph partition solution for small business which runs networking services. More specifically, we are testing the performance (locality) of different partition algorithms under the same condition, in which we treat any replication of data as overhead. The number of servers is set to a range of small values in order to accord with the real world companies with limited resources. 