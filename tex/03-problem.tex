\section{Problem Description}


\subsection{Motivation}

Nowadays, more and more small companies have development software and mobile applications with social networking services. Some of them provides networking function as their main business. While others use it as a method to increase the interaction among their users, and thus attract more potential users of their products. Those small companies need to store their user profiles into the server. However, many startups may encounter problems where they may be lack of budget to deploy adequate number of servers and other equipments. Therefore, they must design their storage algorithm wisely in order to fully utilize their limited resource and achieve a relatively good performance. 

One of the aspects which needs to be considered is to choose a efficient and effective graph partitioning algorithm to separate the user data and deploy them to different servers. It is important because a good deployment of user data at different server can speed up the query access when user would like to obtain it. Small companies needs to provide better user experience in order to survive and thrive. Thus, providing user experience with a fast access rate can help the company to keep and develop its user base[??].  

\subsection{Problem Definition}

Our experiment aims to find a suitable graph partition solution for small business which runs social networking services. More specifically, we are testing the performance of different partition algorithm unders same condition, in which we treat any replication of data as overhead. The number of server is set to a range with small values to accord with the real world companies with limited resources. 
