%\section{Related Work}

      
%\subsection{JA-BE-JA}
%JA-BE-JA  consists of two important components: sampling policy and swapping technique\cite{conf/saso/RahimianPGJH13}. Algorithm IV-B is the core of JA-BE-JA. Node selection is done by the hybrid heuristic. If it fails during the process, random policy start working. A method called FindPartner shows how to select the partner.Then  two sides of a equation is calculated in lines 20 − 25. In the next line, the results are compared. After this, Tr which is the current temperature biases the comparison and lead to do the selection for new states in the first round. Actually the swapping is implemented  to work as an optimistic transaction,So the details are not included here to have better understanding of JA-BE-JA algorithm. The actual swap is done after the two nodes perform a handshake and both of them agree on the swap. This is to avoid  the deciding node holding some outdated information about its partner node. During the step of handshake, the initiating node sends a swap request to the partner node, along with all the information that the partner node needs to verify, such as the current colour (πp), the partner’s colour (πpartner), the number of neighbours with the same colour (dp(πp)), and the number of neighbours with the colour of the partner node (dp(πpartner)). If the verification succeeds, the partner node replies  an acknowledgment (ACK) message to the sender  and the swap will be done successfully. Otherwise, a negative acknowledgment message (NACK) will be sent back and the two nodes preserve their previous colours. These sample and swap processes are periodically repeated by all the nodes.This algorithm will stop until no more swaps happen

%\subsection{Combination of static partitioning and dynamic partitioning}
%Data retrievals of Online social network  require to fetch many small records generated by different users in the network, and the set of records to be retrieved is time dependent due to the location of the data \cite{conf/icde/YuanSCTL12}.Currently, implementation of hash-based partitioning results in accesses at a large number of servers. It  significantly degrades the response time. Also partitioning the OSN with friendship graph is difficult since the power-law degree distribution can lead to a large number of cross-partition edges. Naive replication requires extra storage and the implementation can be very expensive . In 2011,  some researchers came up with a ideal  to partition not only the spatial network of social relations, but also in the time dimension . By implementing this idea, users who have communicated in a certain period can be grouped together and the related information of these users can be put in one server. An activity prediction graph  was generated to keep in one partition to store data that are highly likely to be accessed together. During the research, the  distribution of the Facebook wall posts were analysed in the New Orleans network. The objective of partitioning is to keep the two-hop neighbourhood of a user in one partition, instead of the one-hop network usually considered. Two-hop neighbourhoods are used as the basic units of retrieval in OSN . It is much larger than one-hop networks. It combines a static partitioning method based on KMETIS and a dynamic local partitioning method. The dynamic local partitioning can help maintain evenness and only requires a small amount of data communications among partitions. The results showed that  partitioning on two-hop networks yields $19\%$ more local queries than  one-hop counterpart. Both the static and dynamic methods show several times better data locality than traditional hash-based partitioning. Almost all data movements could be  kept in at most 3 partitions. 
     
%\bigskip


